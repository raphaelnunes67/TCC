% RESUMO--------------------------------------------------------------------------------

\begin{resumo}[RESUMO]
\begin{SingleSpacing}

 Com base na busca da produção de tecnologias para diminuir a degradação do meio ambiente e, unindo-se a isto, a procura pela implementação de \textit{Smart Things}, dentro do âmbito do \textit{IoT - Internet Of Things},  o projeto a ser desenvolvido visa a elaboração de um protótipo de uma \textbf{Cisterna Inteligente - CI} a qual contará com um sistema automático para coleta e distribuição de água da chuva, integrando as três áreas de um projeto \textit{IoT}: \textit{Hardware}, \textit{Firmware} e \textit{Software}. No segmento de \textit{Firmware}, por meio do uso de microcontroladores e microprocessadores em integração com sensores, atuadores e a \textit{Intranet}, será possível coletar dados e realizar ações pré-estabelecidas ou em tempo real oriundas de comandos do usuário, assim como receber atualizações para melhoria contínua do produto (\textit{Firmware Update}).  Na parte de \textit{Software}, a utilização de novas tecnologias para modelagem de aplicações \textit{mobile} e \textit{desktop}, como  \textit{React Native} e \textit{Electron.js},  garantirão a criação de interfaces amigáveis para rotinas de cadastro, atualização, monitoramento e controle dos processos da cisterna. Por fim, na parte de \textit{Hardware}, serão desenvolvidos: placas de circuito impresso, com dispositivos \textit{SMD - Surface Mounted Device}, possuindo todas as estruturas de alimentação, conversão de sinais, comunicação, entre outros; peças estruturais, com base em manufatura aditiva, para a implementação \textit{in loco}.   Edição de teste
 
\vspace{\onelineskip}

\textbf{Palavras-chave}: \textit{IoT - Internet of Things}. \textit{Smart Things}. \textit{Intranet}.  Cisterna Inteligente. \textit{Firmware}. \textit{Hardware}. \textit{Software}. \textit{React Native}. \textit{Electron.js}.

\end{SingleSpacing}
\end{resumo}

% OBSERVAÇÕES---------------------------------------------------------------------------
% Altere o texto inserindo o Resumo do seu trabalho.
% Escolha de 3 a 5 palavras ou termos que descrevam bem o seu trabalho .
% As palavras-chave são separadas por pontos. Apenas a primeira letra é maiúscula.

