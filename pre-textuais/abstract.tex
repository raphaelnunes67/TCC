% ABSTRACT--------------------------------------------------------------------------------

\begin{resumo}[ABSTRACT]
\begin{SingleSpacing}


Based on the search for the production of technologies to reduce the degradation of the environment and, joining this, the trend towards the implementation of Smart Things, within the scope of IoT - Internet Of Things, the developed project aimed at the elaboration of a system to create an Automated Cistern which has a physical prototype for simulating the collection and distribution of rainwater, integrating the three areas of an IoT project: Hardware, Firmware and Software. No Firmware segment, through the use of microcontrollers and microprocessors in integration with sensors, actuators and the Intranet, it will be possible to collect data and perform real-time actions from user commands, as well as updates for continuous improvement of the Embedded Software (Firmware To update). In the Software section, the use of technologies for modeling Mobile and Desktop applications, such as React Native and Electron, ensured the creation of user-friendly interfaces for routines for registration, updating, monitoring and control of collection system processes. Finally, in the Hardware section, schematics were created, enabling the creation of layouts, which will later guarantee the manufacture of printed circuit boards - PCB's. With the purpose of validation, the built prototype has structures for: power, signal conversion, communication, control and activation.

%3D schematics

\vspace{\onelineskip}

\textbf{Keywords}: \textit{IoT - Internet of Things}. \textit{Smart Things}. \textit{Intranet}.   \textit{Firmware}. \textit{Hardware}. \textit{Software}. \textit{React Native}. \textit{Electron}.

\end{SingleSpacing}
\end{resumo}

% OBSERVAÇÕES---------------------------------------------------------------------------
% Altere o texto inserindo o Abstract do seu trabalho.
% Escolha de 3 a 5 palavras ou termos que descrevam bem o seu trabalho 
% As palavras-chave são separadas por pontos. Apenas a primeira letra é maiúscula.