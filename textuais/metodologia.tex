% METODOLOGIA------------------------------------------------------------------

\chapter{METODOLOGIA}
\label{chap:metodologia}


\section{Análise e definição das características do projeto}

Para a metodologia deste trabalho de conclusão de curso, partiu-se do princípio de uma estrutura previamente construída, demostrada no esquema abaixo. A partir dos elementos dessa figura, evidenciados na \autoref{tab:tabela_esquema_cisterna}, foi possível realizar uma análise sobre o que pode ser automatizado, levando em consideração as variáveis e características do sistema.  

\begin{figure}[H]
	\centering
	\caption{Esquema de demonstração de uma cisterna no subsolo}
	\includegraphics[width=1.0\textwidth]{figuras/esquema_cisterna.png}
	\fonte{Adaptado (ECOMONTES, 2016)}
	\label{fig:esquema_cisterna}
\end{figure}

\newpage

\begin{table}[]
	\centering
	\small
	\begin{tabular}{c|c|c}
		\hline
		\textbf{Identificador} & \textbf{Elemento} & \textbf{Descrição} \\ \hline
		1 & Calha coletora & \begin{tabular}[c]{@{}c@{}}Elemento convencional para coleta e \\ descarte de água da chuva\end{tabular} \\ \hline
		2 & Filtro A (cascalho fino) & \begin{tabular}[c]{@{}c@{}}Elemento para realização de \\ filtragem de pequenas impurezas\end{tabular} \\ \hline
		3 & Filtro B (cascalho grosso) & \begin{tabular}[c]{@{}c@{}}Elemento para realização de \\ filtragem de impurezas\end{tabular} \\ \hline
		4 & Tubulação de descarte & \begin{tabular}[c]{@{}c@{}}Tubulação utilizada como rota de escoamento \\ quando o reservatório não está em uso ou \\ está cheio\end{tabular} \\ \hline
		5 & Reservatório de coleta & Cisterna propriamente dita \\ \hline
		6 & Motobomba ou bomda d'água & \begin{tabular}[c]{@{}c@{}}Elemento utilizado para realização do \\ ganho de elevação da água\end{tabular} \\ \hline
		7 & Caixa d'água auxiliar & \begin{tabular}[c]{@{}c@{}}Caixa d'água convencional com alimentação \\ oriunda da bomba d'água\end{tabular} \\ \hline
		8 & Caixa d'água convencional & \begin{tabular}[c]{@{}c@{}}Caixa d'água padrão com alimentação \\ da estação de água da cidade\end{tabular} \\ \hline
		9 & Elo de ligação & \begin{tabular}[c]{@{}c@{}}Ligação utilizada para abastecer a caixa d'água \\ auxiliar quando a cisterna está seca \\ ou em manutenção\end{tabular} \\ \hline
		10 & Distribuidor & \begin{tabular}[c]{@{}c@{}}Elementos de distribuição de água \\ para pontos estratégicos\end{tabular} \\ \hline
	\end{tabular}
	\caption{Identificação dos elementos da \autoref{fig:esquema_cisterna}.}
	\label{tab:tabela_esquema_cisterna}
\end{table}
% https://www.tablesgenerator.com

Com base nesses elementos é importante destacar os seguintes pontos para as implementações que serão tratadas nas seções posteriores:

\begin{itemize}
	\item Definir o método ou sistema para medição de nível do item 5;
	\item Elaborar o sistema de ativação/desativação da motobomba do item 6;
	\item Definir o método ou sistema para medição de nível do item 7;
	\item Elaborar um sistema para intermediar/controlar a passagem de água no elo de ligação (item 9);
	\item Definir um sistema para administrar o descarte de água da cisterna (item 4);
	\item Esquematizar e situar os sistemas para envio e aquisição de dados, tornando-se possível realizar remotamente as operações citadas nos itens anteriores; 
	\item Configurar um pequeno servidor para intermediar entre os dispositivos microcontrolados e as interfaces de controle;  
	\item Implementar as interfaces para visualizar/comandar os pontos citados nos itens anteriores.
\end{itemize}

Dois módulos distintos foram propostos: \textbf{\textit{Tank Control Module} - TCM}, o qual será responsável pelo monitoramento e controle da caixa d'água auxiliar (\autoref{fig:esquema_cisterna}, identificador 7) e \textbf{\textit{Cistern Control Module} - CCM}, responsável pelo monitoramento e controle da cisterna (\autoref{fig:esquema_cisterna}, identificador 5). O diagrama da \autoref{fig:esquema_proj} mostra o esquema proposto para a criação do projeto. 

\begin{figure}[H]
	\centering
	\caption{Esquema básico do projeto.}
	\includegraphics[width=1.05\textwidth]{figuras/esquema_basico_proj_2.png}
	\fonte{Própria}
	\label{fig:esquema_proj}
\end{figure} 

\section{Organização do trabalho}

Iniciou-se com a criação do quadro \textit{Kanban} (\autoref{fig:kanban-proj}), para auxílio da organização das tarefas, e com a criação do repositório no \textit{Github} (\autoref{fig:github}) para armazenamento e versionamento dos códigos das vertentes de \textit{Firmware} e \textit{Software} do projeto. Deu-se prosseguimento com os estudos e levantamentos bibliográficos relacionando três áreas do projeto. Buscou-se encontrar os métodos,  ferramentas, tecnologias, bibliotecas e \textit{frameworks} mais adequados para a implementação do projeto.  Diante de cada seleção feita, foram executadas análises para que fosse definido o funcionamento do sistema total, munido da combinação de cada uma das três áreas citadas anteriormente e visando a geração de um produto que pudesse ser empregado no mercado: atrativo economicamente e seguindo diretrizes sustentáveis.  

Primeiramente, na seção de \textit{hardware}, foram selecionados as ferramentas para modelagem de placas de circuito impresso - \textit{PCB's} e para modelagem de peças \textit{3D}, através de manufatura aditiva, a escolha de todos os dispositivos eletrônicos a serem utilizados e a análise do local de aplicação. Posteriormente, na parte de \textit{firmware}, já estando selecionados os microcontroladores e o microprocessador, foram determinadas todas as rotinas de operação e escolhidas, respectivamente, as linguagens para programá-los e o \textit{framework} para criação de um sistema operacional embarcado baseado em \textit{kernel Linux}. Na parte de \textit{software}, foram selecionadas as ferramentas para a criação de interfaces dentro da camada \textit{front-end}: aplicações \textit{mobile} e \textit{desktop}.

Por fim, elaborou-se a lista de materiais necessários para construção o projeto, tendo objetivo de validação em ambiente real, efetuando testes de longos períodos, averiguando a integridade do sistema, a robustez dos componentes e a identificação de casos não previstos anteriormente, tornando possível a aplicação de melhorias posteriores a entrega deste trabalho.

\begin{figure}[H]
	\centering
	\caption{Visão geral do quadro \textit{Kanban} criado na ferramenta \textit{Trello}}
	\includegraphics[width=0.85\textwidth]{figuras/visão_geral_meu_kanban.png}
	\fonte{Própria.}
	\label{fig:kanban-proj}
\end{figure}


\begin{figure}[H]
	\centering
	\caption{Visão geral repositório criado no \textit{Github}}
	\includegraphics[width=0.85\textwidth]{figuras/github.png}
	\fonte{Própria.}
	\label{fig:github}
\end{figure}



\section{Levantamento do referencial bibliográfico e capacitação}
\label{sec:metmodal}

Essa parte do trabalho conteve-se no levantamento de referências bibliográficas relacionadas com os temas de \textit{IoT}, programação de microcontroladores, criação de aplicações com \textit{frameworks} baseados em \textit{Javascript}. Também buscou-se a capacitação em cursos oferecidos pelas plataformas \textit{Alura}, \textit{Udemy} e \textit{Skylab (Rocketseat)} o aprendizado de conteúdos complementares ao curso de formação em engenheria de controle e automação, como a criação de placas de circuito impresso, treinamentos sobre \textit{Linux} embarcado, implementação do protocolo \textit{MQTT}, criação de \textit{APPs Android} com \textit{React Native} e aplicações \textit{Desktop} com \textit{Electron.js}.

\section{Desenvolvimento dos elementos de \textit{Hardware}}
\label{sec: dev_ele_hw}

Nesse momento foram definidos todos os elementos de \textit{hardware} necessários para a execução do projeto: para os dispositivos eletrônicos, documentos como \textit{Datasheets}, catálogos e informativos de dispositivos elétricos foram coletados para consulta durante o decorrer do trabalho; para a parte estrutural executou-se a enumeração de peças que deveriam o desenhadas com auxílio do \textit{Software Autodesk Inventor} e posteriormente construídas através de manufatura aditiva.  Realizou-se uma busca no mercado pelos componentes necessários efetuando as possíveis compras e elaborando o orçamento geral do projeto.

\subsection{O circuito de alimentação}

Nesta etapa do projeto iniciou-se a criação do circuito de alimentação para os módulos \textbf{CCM} e \textbf{TCM}. Partindo-se da fonte chaveada definida no \autoref{chap:fundamentacao-teorica}, a qual transforma 110V AC em 12V DC, criou-se um conversor DC-DC do tipo \textit{Buck} para converter a tensão de saída da fonte para 3.3V, ideais para alimentação dos sensores e microcontroladores. 

\begin{figure}[H]
	\centering
	\caption{Esquma de ligação do circuito de alimentação}
	\includegraphics[width=0.8\textwidth]{figuras/alimentacao.png}
	\fonte{Própria.}
	\label{fig:alimentacao_esquema}
\end{figure}

Após a definição do esquema de medição, procurou-se dimensionar o conversor \textit{Buck}. O dimensionamento deu-se a partir das leituras dos \textit{datasheets} de todos os dispositivos que futuramente poderiam ser utilizados. Outro ponto importante salientar para a escolha do conversor \textit{Buck} foi a disponibilidade no mercado.

Dentre os fatores citados o conversor selecionado foi o LM2596, o qual podemos encontrar um módulo com sua aplicação típica (\autoref{fig:conversor_buck}) com certa facilidade no mercado. A \autoref{fig:conversor_buck_teste} mostra o teste realizado em bancada.

\begin{figure}[H]
	\centering
	\caption{Aplicação típica do LM2596.}
	\includegraphics[width=0.8\textwidth]{figuras/conversor_buck.jpg}
	\fonte{Própria.}
	\label{fig:conversor_buck}
\end{figure}

\begin{figure}[H]
	\centering
	\caption{Teste em bancada do módulo com LM2596.}
	\includegraphics[width=0.7\textwidth]{figuras/conversor_buck_teste.jpg}
	\fonte{Própria.}
	\label{fig:conversor_buck_teste}
\end{figure}

\subsection{O circuito de detecção de fechadura da chave}
\subsection{O circuito de medição de nível}
\subsection{O circuito de ativação da motobomba}
\subsection{O circuito de ativação da válvula}

\section{Desenvolvimento dos elementos de \textit{Firmware}}
\label{sec: dev_ele_fw}

A definição e desenvolvimento dos elementos de \textit{Firmware} se deu após todos os levantamentos de requisitos do trabalho. Esse momento concentrou-se na escolha de tecnologias mais fáceis de implementação, que tivessem uma gama de documentações e que fossem empregadas em produtos oficiais. Para lidar com o uso dos microprocessadores e microcontroladores, foram feitas pesquisas visando consolidar conceitos aprendidos durante o curso de graduação.

\subsection {Instalação e configuração do \textit{Broker MQTT}}
\subsection {Programação do servidor}
\subsection {Definição de utilização dos pinos do ESP-12S}
\subsection{Organização dos arquivos internos dos microcontroladores}
\subsection {Implementação do \textit{Driver WIFI}}
\subsection {Implementação do \textit{Driver MQTT}}
\subsection{Implementação da rotina de Reset}
\subsection{Implementação da funcionalidade de \textit{OTA Upgrade}}
\subsection{Implementação do modo de \textit{DeepSleep}}


\section{Desenvolvimento dos elementos de \textit{Software}}


Para o desenvolvimento dos elementos de \textit{Software} buscou-se a participação de cursos básicos e avançados sobre aplicações \textit{front-end}, as quais servem de conteúdo complementar aos temas abordados na graduação em engenharia de controle e automação.

Os cursos adquiridos trouxeram uma série de conhecimentos para uma maior interação entre desenvolvedor e usuário, possibilitando a criação de interfaces intuitivas e agregando novas informações e visões a outros temas, como na programação de sistemas embarcados.

\section{Desenvolvimento da aplicação \textit{Desktop}}
\section{Desenvolvimento da aplicação \textit{Mobile}}



\subsection{Os elementos de ativação e desativação}

Tendo em vista os elementos descritos nos itens anteriores, tornou-se necessário a implementação de um circuito para ativação e desativação da bomba d'água assim como a execução de técnicas de proteção e isolamento.

O diagrama desenvolvido no \textit{software Proteus} (\autoref{fig:diagramaproteus}) mostra a integração de como seria o sistema de ativação e desativação da motobomba.

\begin{figure}[H]
	\centering
	\caption{Diagrama de ativação com partida lenta}
	\includegraphics[width=0.9\textwidth]{figuras/diagrama_ativação_bomba.png}
	\fonte{Própria}
	\label{fig:diagramaproteus}
\end{figure} 

A partir deste diagrama utilizou-se técnicas de partida lenta através do chaveamento transistorizado (\textit{BC546}). Por meio do \textit{PWM - Pulse-Width Modulation} oriundo do microcontrolador, o transistor realiza a alteração sobre o valor eficaz de tensão aplicada na bomba d'água.

Para saturação do transistor \textit{IRF1404} (\autoref{fig:IRF1404}) foi desenvolvido o circuito dobrador de tensão também evidenciado na \autoref{fig:diagramaproteus}. A ideia desse circuito é garantir uma queda de tensão entre \textit{GATE} e \textit{SOURCE} duas vezes maior (em torno de \textit{24V}) que a tensão de alimentação do circuito.

\begin{figure}[H]
	\centering
	\caption{Transistor IRF1404}
	\includegraphics[width=0.2\textwidth]{figuras/IRF1404.png}
	\fonte{ALLDATASHEET, 2020}
	\label{fig:IRF1404}
\end{figure} 
