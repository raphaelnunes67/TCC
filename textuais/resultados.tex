% RESULTADOS-------------------------------------------------------------------

\chapter{RESULTADOS}
\label{chap:resultados}

%Cada capítulo deve conter uma pequena introdução (tipicamente, um ou dois parágrafos) que deve deixar claro o objetivo e o que será discutido no capítulo, bem como a organização do capítulo. 

Após a formulação e teste de cada item dos sistemas deu-se início a junção de todas as funcionalidades arquitetadas. O conteúdo deste capítulo visa apresentar os resultados obtidos neste trabalho em cada área de desenvolvimento.

Primeiramente como resultado da parte de \textit{hardware} podemos apresentar a formulação dos circuitos dos módulos \textbf{CCM} e \textbf{TCM} apresentados nas figuras abaixo.

Os dois módulos foram montados em \textit{protoboard} buscando validar, em conjunto, todas funcionalidades as seções descritas na metodologia.

No quesito de \textit{firmware}, alcançou-se todas as funcionalidades propostas, gerando um código estável e escalável para projetos reais.

Por fim, também obtivemos como resultado a elaboração de telas utilizando as tecnologias propostas, atingindo o objetivo de relacionar o \textit{software} com as camadas de \textit{firmware} e \textit{hardware}. 