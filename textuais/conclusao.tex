% CONCLUSÃO--------------------------------------------------------------------

\chapter{CONCLUSÃO}
\label{chap:conclusao}

Este trabalho de conclusão de curso teve como principal objetivo desenvolver um projeto base para aplicação de um sistema automatizado para coleta, armazenamento e distribuição de água da chuva tendo como foco atuar sobre os processos convencionais de uma cisterna. Partindo da estratégia de reunir diversas ferramentas, metodologias e tecnologias estudadas durante o curso de engenharia de controle e automação, houve a possibilidade de que fossem estabelecidos conceitos a serem aplicados em um projeto real.

Para as partes físicas do projeto (\textit{hardware}) considerou-se uma cisterna convencional (não automatizada). A partir de uma esquematização foi possível estabelecer os pontos principais para se aplicar uma certa automação, propondo os módulos \textbf{CCM}, atrelado à cisterna e \textbf{TCM}, atrelado ao reservatório auxiliar.

Durante o desenvolvimento dos módulos, diversos testes de conceito foram aplicados. Em ambos os módulos foram propostas implementações com válvula solenoides, para alternar o fluxo de água, e medição de nível por meio de sensores ultrassônicos. Individualmente, no \textbf{CCM} foi determinada a utilização de uma motobomba \textit{DC} com o planejamento de todo seu circuito de controle e no \textbf{TCM} estruturou-se um sistema de segurança utilizando chave do tipo boia (também conhecida como chave de flutuação).

O projeto também englobou elementos de \textit{software} buscando integrações com os ambientes \textit{desktop} e \textit{mobile}. Os \textit{frameworks} \textit{Electron} e \textit{React} ajudaram a criar moldes de aplicações mais interessantes para um usuário final, possibilitando o controle dos componentes do sistema de forma mais intuitiva.

Com isso, tomando como base os conceitos de Internet das Coisas e utilizando as ferramentas atuais para desenvolvimento de \textit{firmware} e \textit{software} foi-se possível criar um projeto base para automatizar as operações de uma cisterna pluvial.

\section{Trabalhos Futuros}
\label{sec:trabalhosFuturos}

Neste item destaca-se alguns pontos que podem ser implementados ou melhorados baseados no tema abordado:

\begin{itemize}
\item Criar uma rotina para informar as dimensões dos tanques os quais se desejar medir o volume;
\item Adicionar a funcionalidade de controle automático com base nas variáveis do sistema (como, por exemplo, o volume do tanque) e de acordo um horário estabelecido;
\item Criar uma forma de autenticação ao repositório de novos binários para atualização de \textit{firmware};
\item Confeccionar a
s placas de circuito impresso (\textit{PCB's}) e aplica-las à um sistema real;
\item Implementar ou definir as peças necessárias para fixar os sensores, conectores e placas dos módulos; 
\item Gerar os arquivos de distribuição da aplicação \textit{desktop};
\item Gerar os arquivos de distribuição da aplicação \textit{mobile};
\end{itemize}
