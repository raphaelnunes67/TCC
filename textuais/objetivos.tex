% OBJETIVOS-------------------------------------------------------------------

\chapter{OBJETIVOS}
\label{chap:objetivos}

\section{Objetivo geral}
\label{sec:objetivogeral}
O objetivo desse trabalho de conclusão de curso é a elaboração e aplicação de um sistema automatizado de coleta, armazenamento e distribuição de água da chuva com base em uma cisterna pluvial. Realizando medições de volume, através de sensores, acionamento de válvulas solenoides e motobombas por meio de comandos via interfaces desktop e mobile, o projeto também contará com a elaboração de placas de circuito impresso - \textit{PCB's}, o uso de microcontoladores e microprocessadores em conjunto com  diretrizes de Internet das Coisas - \textit{IoT},  assim como o uso de \textit{Frameworks} atuais baseados em \textit{Javascript} para criação de interfaces Homem-Máquina. 

\section{Objetivos específicos}

\begin{description}
	\item [(a)] Elaborar um módulo denomiado \textbf{SMCC - Sistema de Medição e Controle da Cisterna} para aplicação no reservatório principal, realizando medições de nível, acionamento de \textit{motobomba}, direcionamento do fluxo de água e possuindo conexão com outros dispositivos via \textit{ Wi-Fi}.
	\item [(b)] Elaborar um módulo denominado \textbf{SMCCD - Sistema de Medição e Controle da Caixa D'água} para aplicação no reservatório auxiliar, realizando medições de nível, direcionamento do fluxo de água e possuindo conexão com outros dispositivos via \textit{Wi-Fi}.
	\item [(c)] Elaborar um aplicação \textit{Android Mobile} (denominada \textbf{SCCP APP}) e \textit{Desktop} (denominada \textbf{SCCP DESKTOP}) para executar as ações: ativação e desativação de uma bomba d'água, direcionamento do fluxo de água, visualização de dados provenientes de sensores e a possibilidade de definir as condições que executem rotinas de acionamento automático;
	\item [(d)] Elaborar um sistema operacional embarcado conciso (baseado em \textit{kernel Linux}) aplicando-o a um microprocessador. O dispositivo deve possuir conexão com a \textit{Intranet} e servir como uma central de controle e armazenamento de dados, bem como ser hospedeiro do serviço de \textit{Broker MQTT};
	\item [(e)] Incluir no sistema, rotinas de calibração para configurar os níveis e estados de operação em cisternas e caixas d'águas de tamanhos e formatos diferentes.
	\item [(f)] Organizar um repositório \textit{online} para realização de futuros \textit{updates} ou \textit{upgrades} do \textit{firmware} embarcado;
	\item [(g)] Planejar a implementação \textit{in loco},  analisando todos os fatores estruturais. Adqurir ou confeccionar instrumentos para proteção e suporte dos itens citados nos quesitos \textit{(a)} e \textit{(b)}.
\end{description}

\label{sec:objetivosespecificos}