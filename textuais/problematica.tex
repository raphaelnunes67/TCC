% PROBLEMÁTICA-------------------------------------------------------------------

\chapter{PROBLEMÁTICA}
\label{chap:problematica}

Segundo a Gartner, Inc BizMeet \cite{BizMeet} desde 2017 existem mais objetos conectados à Internet do que as 7 bilhões de pessoas no mundo. Isso demonstra uma crescente busca pela obtenção e controle das mais diversas tarefas. Os objetos e dispositivos estão adquirindo funcionalidades as quais se identificam com os ramos da \textit{IoT}, tornando-se possível a concepção de diversas melhorias no funcionamento, através da obtenção e distribuição de dados.  A utilização da \textbf{Internet das Coisas} está tão conectada ao nosso cotidiano que ocasiona o nascimento de tecnologias necessariamente conectadas, como \textit{Smart Home} e \textit{Smart Buildings}. 

Diante disso, em ambientes como na região Amazônica, onde há altos índices de precipitação, a implantação de cisternas para utilização da água da chuva se torna bastante viável. Sabe-se que a água da chuva não é própria para o consumo e para o preparo de alimentos, porém a mesma pode ser utilizada em afazeres como limpeza de locais e também em descargas de banheiros, onde há o maior nível de desperdício.

Em virtude do que foi mencionado,  a aplicação de uma cisterna se torna algo viável para a economia de água ajudando na conservação do meio ambiente, porém, é possível elaborar um projeto base para ser aplicado a diversas situações? Quais ferramentas são necessárias para realizar as medições de nível? Como fazer um sistema o qual os dados poderão ser acessados remotamente? Quais cuidados se deve ter ao implantar tal sistema?
