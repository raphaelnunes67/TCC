% PROBLEMÁTICA-------------------------------------------------------------------

\chapter{PROBLEMÁTICA}
\label{chap:problematica}

Segundo a Gartner, Inc BizMeet \cite{BizMeet} desde 2017 existem mais objetos conectados à Internet do que as 7 bilhões de pessoas no mundo. Isso demonstra uma crescente busca pela obtenção e controle das mais diversas tarefas. Os objetos e dispositivos estão adquirindo funcionalidades as quais se identificam com os ramos da \textit{IoT}, tornando-se possível a concepção de diversas melhorias no funcionamento, através da obtenção e distribuição de dados.  A utilização da \textbf{Internet das Coisas} está tão conectada ao nosso cotidiano que ocasiona o nascimento de tecnologias necessariamente conectadas, como \textit{Smart Home} e \textit{Smart Buildings}. 

Diante disso, em ambientes como na região Amazônica, onde há altos índices de precipitação, a implantação de cisternas para utilização da água da chuva se torna bastante viável. Sabe-se que a água da chuva não é própria para o consumo e para o preparo de alimentos, porém a mesma pode ser utilizada em afazeres como limpeza de locais e também em descargas de banheiros, onde há o maior nível de desperdício.

Em virtude do que foi mencionado,  a aplicação de uma cisterna se torna algo viável para a economia de água ajudando na conservação do meio ambiente, porém, a automação do processo de coleta e distribuição trará reais vantagens quando comparada à um processo manual? Quais serão os problemas recorrentes ao se implementar tal sistema?  Será possível uma aplicação em diversos contextos e localidades?  O auxílio de tecnologias como o \textit{WIFI} unido à \textit{IoT} fará com que a aplicação seja mais vantajosa?  A utilização da água coletada por meio de uma cisterna automatizada, onde haverá gastos, como consumo energia elétrica, aplicação de sensoriamento, motobombas, sistemas hidráulicos, entre outros, trará benefícios financeiros quando comparado à utilização de água através, unicamente, da rede tradicional?
