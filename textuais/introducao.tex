% INTRODUÇÃO-------------------------------------------------------------------

\chapter{INTRODUÇÃO}
\label{chap:introducao}


A água é um recurso básico para a sustentação humana em um ambiente e ao longo do tempo várias civilizações evoluíram e padeceram em função de sua relação de uso com este recurso. Nos últimos anos, crises relacionadas ao abastecimento e à qualidade da água potável têm sido observadas em todo o Globo. Tomando o Brasil como exemplo, percebe-se uma distribuição desigual: No norte do pais há grandes reservas de água, porém nas regiões Nordeste e Sudeste há problemas de escassez e poluição dos rios.


A região norte, possuindo a maior reserva de água potável do Brasil é, segundo a Globo, a região que possui índices mais altos de desperdício. O desperdício pode ser encontrado no ambiente doméstico, falta de orientação e nas várias etapas do processo: coleta, armazenamento, processamento e finalidade final do recurso. 

A aplicação de uma cisterna automatizada garante uma supervisão do nível de água constantemente assim como o controle/acionamento de bombas para alimentação de tanques ou caixas d’agua para tarefas específicas trazendo comodidade e fomentando os motivos de aplicação. O aproveitamento pode contribuir muito também para a redução da
incidência de inundações nas grandes cidades decorrentes do volume excessivo de chuvas em
determinadas épocas do ano. Isso porque o excesso de áreas impermeabilizadas nos ambientes
urbanos, é um dos fatores preponderantes para ocorrência de enchentes, pois a água da chuva
não tem como infiltrar no solo e chegar ao lençol freático, acumulando-se nas ruas e
transbordando nos rios. Assim, a implantação de sistemas de reaproveitamento dessas águas
iria reduzir sensivelmente esse problema.