% INTRODUÇÃO-------------------------------------------------------------------

\chapter{INTRODUÇÃO}
\label{chap:introducao}


A água é um recurso básico para a sustentação humana em um ambiente e ao longo do tempo várias civilizações evoluíram e padeceram em função de sua relação de uso com este recurso. Nos últimos anos, crises relacionadas ao abastecimento e à qualidade da água potável têm sido observadas em todo o Globo. Tomando o Brasil como exemplo, percebe-se uma distribuição desigual: No norte do pais há grandes reservas de água, porém nas regiões Nordeste e Sudeste há problemas de escassez e poluição dos rios.


A região norte, possuindo a maior reserva de água potável do Brasil e também os maiores índices de precipitação, é a região que possui as taxas mais altas de desperdício \cite{globo}. O desperdício pode ser encontrado no ambiente doméstico e nas várias etapas de: coleta, armazenamento, processamento e principalmente na distribuição do recurso. 

A instalação de uma cisterna garante, pelo menos, três pontos positivos: que seja possível utilizar a água de precipitações para afazeres domésticos, reduzindo o consumo mensal de determinada residência; pode diminuir o desperdício durante a etapa de distribuição da concessionária e contribuir para a redução da
incidência de inundações em grandes cidades, uma vez que grande parte dessa água não seria descartada, mas armazenada.

A aplicação de uma cisterna automatizada garante uma supervisão do nível de água constantemente assim como o controle/acionamento de bombas para alimentação de tanques ou caixas d’agua para tarefas específicas trazendo comodidade e fomentando os motivos de aplicação.