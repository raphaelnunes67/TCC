% HIPÓTESE E JUSTIFICATIVA-------------------------------------------------------------------

\chapter{HIPÓTESE E JUSTIFICATIVA}
\label{chap:hipoteseejustificativa}

\section{Hipótese}
\label{sec:hipotese}

A inclusão da \textit{IoT} nos mais diversos processos nos traz múltiplos benefícios, como: obter maiores informações e atuar sobre tais processos. Automatizar tarefas que são, muitas vezes repetitivas ou mesmo insalubres faz com que as tecnologias denominadas \textit{Smart} se tornem cada vez mais um foco de pesquisa e desenvolvimento. 

No âmbito do meio ambiente, a implantação de uma cisterna garante um menor desperdício de água, utilizando-a para tarefas específicas assim como uma economia no consumo. Porém a implantação de uma cisterna como não atrai indivíduos pois existe a necessidade de trabalho braçal assim como a constante análise da quantidade de água disponível.

Esse trabalho de conclusão de curso busca elaborar um projeto base para a aplicação de uma cisterna automatizada, situando as tecnologias, os dispositivos e as ferramentas necessárias para executar o controle por meio de \textit{smartphones} e computadores. Tais tecnologias poderão tornar a ideia de aplicação mais atraente para possíveis investimentos, em caso do projeto se tornar base de um produto.



\section{Justificativa}
\label{sec:justificativa}

Buscando estar de acordo com a convergência de produtos e processos conectados em rede, percebeu-se a necessidade e oportunidade da utilização de conceitos e tecnologias emergentes nos contextos atuais, como \textit{Smart Things}. A aplicação de serviços que unam a busca pela preservação ambiental e conceitos citados anteriormente se destacam por possibilitar uma gama de vantagens em qualquer processo.

O presente trabalho se justifica pelos altos índices de precipitação na região do Estado do Amazonas, onde será aplicado. A coleta de água da chuva acarretará, além da diminuição do consumo de água (trazendo economia para o usuário e ajudando na preservação ambiental), em uma distribuição mais inteligente da água não potável: que pode ser utilizada para limpezas, descargas e até irrigações; e em uma motivação para investimentos futuros reduzindo, ainda mais, os gastos de implementação.

O projeto trará motivações e investidores para aplicação de novas pesquisas focadas em contextos diferentes dentro do estado, onde há: \textbf{I}. lugares que não possuem estruturas de rede ou que demandam comunicação à grandes distâncias: utilização de tecnologias como \textit{LoRaWan}, \textit{SigFox} e \textit{Zigbee}; \textbf{II.} lugares que não possuem energia elétrica: união dos conceitos desse trabalho com fontes renováveis, como alimentação fotovoltaica.
